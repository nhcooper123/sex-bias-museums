%-------------------------------------------------------------------------------
% Preamble
%-------------------------------------------------------------------------------

\documentclass[a4paper, 12pt]{article}
\usepackage{ms}            % load the template
\usepackage[osf]{mathpazo} % palatino
%\usepackage[round]{natbib} % author-year citations
\usepackage[superscript,biblabel]{cite} % for superscript citations
\usepackage{graphicx}
\usepackage{subcaption}
\usepackage{parskip} 
\usepackage{amsmath}
\usepackage{longtable}
\usepackage{pdflscape}

\pagenumbering{arabic}  
\linespread{1.66}

%-------------------------------------------------------------------------------
% Title page information
%-------------------------------------------------------------------------------

\title{Sex biases in bird and mammal natural history collections}

\author{
  Natalie Cooper$^{1*}$, 
  Alexander L. Bond$^{1,2}$,
  Kristofer M. Helgen$^{3}$, \\
  Roberto Portela Miguez$^{1,4}$, and
  Louise Tomsett$^{1,4}$
}
\date{}
\affiliation{\noindent{\footnotesize
  $^1$Department of Life Sciences, Natural History Museum, Cromwell Road, London, SW7 5BD, UK.\\
  $^2$Bird Group, Department of Life Sciences, Natural History Museum, Akeman Street, Tring, Hertfordshire, HP23 6AP, UK.\\ 
  $^3$Department of Ecology and Evolutionary Biology, School of Biological Sciences, University of Adelaide, North Terrace,  Adelaide, SA 5005, Australia.\\
  $^4$Mammal Group, Department of Life Sciences, Natural History Museum, Cromwell Road, London, SW7 5BD, UK.\\
  $*$Email address: natalie.cooper@nhm.ac.uk
}}

\vfill

\runninghead{Sex biases in collections}
%\keywords{}
%}

%-------------------------------------------------------------------------------
% Begin document
%-------------------------------------------------------------------------------
\begin{document}
\modulolinenumbers[1]   % Line numbering on every line

\mstitlepage

\parindent = 1.5em
\addtolength{\parskip}{.9em}

\raggedright
%-------------------------------------------------------------------------------
% Abstract
%-------------------------------------------------------------------------------

\section{Abstract}

Natural history specimens are used extensively in studies of taxonomy, systematics, genomics, comparative anatomy, development, parasitology,
toxicology, morphological evolution, anthropogenic change and more.
Although biological sex may influence all of these areas, it is often overlooked in large-scale studies using specimens. 
If collections are biased towards one sex, studies may not be representative of the species. 
Here, we investigate sex ratios in over two million bird and mammal specimen records from five large international museums. 
We found a slight bias towards males in birds (40\% females) and mammals (48\% females), but this varied among orders. 
The proportion of female specimens has not significantly changed in 130 years, but decreased in species with “showy” male traits like colorful plumage and horns. 
Body size had little effect. 
Male bias was strongest in name-bearing types; only 27\% of bird and 39\% of mammal types were female. 
These results imply that previous studies may be impacted by undetected male bias, and vigilance is required when using specimen data, collecting new specimens, and designating types.

\textbf{Keywords: sex bias, museum specimens, natural history collections, birds, mammals}

%-------------------------------------------------------------------------------
% Main text
%-------------------------------------------------------------------------------

\section{Introduction}\label{main}
Natural history museum specimens are used extensively in studies of taxonomy, systematics, biogeography, genomics, comparative anatomy, morphological variability, development, parasitology, stable isotope ecology, toxicology, morphological evolution and more\cite{lister2011natural,pyke2010biological,mclean2015natural}. 
They are also of vital importance for understanding Anthropocene effects on biodiversity\cite{meineke2018biological}. 
Large studies of species phenotypes using museum specimens, especially in vertebrates, are becoming increasingly common (e.g.\cite{cooney2017mega,felice2018developmental}) and are revealing new insights into the evolution of diversity. 
These studies require large amounts of data, therefore the focus is often on collecting data from as many species as possible, to the detriment of other sources of variation. 
Sex is an important factor that influences many aspects of an individual's ecology and life-history (Table \ref{table_consequences}), but it is often treated as a nuisance variable, overlooked entirely, or data collection focuses on just one sex (e.g.\cite{cooper2009factors}) to avoid the issue. 
If natural history collections have unbiased sex ratios (i.e. close to 50\% males and females, or reflective of the sex ratio for the species is in the wild\cite{karlin1986theoretical}) then this may not be a problem; if there is a bias in the sex composition of collections, this has implications for studies that assume their samples are representative of the whole population or species (Table \ref{table_consequences}). 
No large-scale study of sex ratios in museum collections exists, therefore investigating this is of vital importance as the number of studies using museum specimens continue to rise (e.g. this recent special issue\cite{meineke2018biological}).

\newpage
\begin{landscape}
  % Table 1

\begin{longtable}{p{3cm} p{16cm}}

\caption{Prominent uses of natural history specimens and how research outcomes may be influenced by sex biases.}\\ 
  
  % Header
  \hline
  \textbf{Use} & \textbf{Might sex biases in birds and mammals affect research outcomes?}\\ 

  % Body of table
  \hline
  Taxonomy & \textbf{Yes}. Sexes often have external differences; if these are used in the taxonomy of the group (e.g. male plumage colouration in birds\cite{paxton2009utility}) then taxonomy may be harder in one sex than another. If sexes are very different, and there is strong sex bias in name-bearing types, this may result in taxonomic inflation, with males and females being given different names (although this is nowadays rare).\\ 

  Systematics & \textbf{Maybe}. For standard molecular phylogenies, commonly used genes do not differ substantially among sexes (i.e. not to the extent that they would form different branches). In phylogenomic studies, however, gene trees may vary across a genome if sex chromosomes are included in the sample\cite{reddy2017}. Morphological phylogenies are likely to be most affected, as morphological characters can vary extensively between males and females. This also has implications for Total Evidence phylogenies that use both morphological and molecular data.\\ 

  Biogeography & \textbf{No}. Locally sexes may be spatially segregated (e.g. bat roosting sites\cite{altringham}), and have different dispersal rates\cite{pusey1987sex}, but sexes (necessarily) do not differ in terms of large scale biogeography.\\ 

  Genomics & \textbf{Yes}. Mammals and birds have chromosomal sex determination; in mammals XY male and XX female, in birds ZZ male and ZW female\cite{stevens1997sex}. The X and Z chromosomes are larger and have more genes than W and Y, thus genome size differs among sexes. Many genes are also sex-linked, so genomes will differ among sexes.\\ 

  Comparative anatomy & \textbf{Yes}. Males and females have internal and external anatomical differences, thus sex biases will influence comparative anatomy studies.\\ 

  Development & \textbf{Maybe}. In most vertebrates, early developmental stages are almost identical in males and females, however later development and sexual maturation involve highly divergent growth to result in adult sex differences \cite{badyaev2002growing}. If research is focused on early development or juvenile life-history stages then sex biases are unlikely to pose a problem.\\

  Morphological variability & \textbf{Maybe}. Perceived wisdom is that males are more variable than females. However, many detailed morphometric studies do not find this (e.g.\cite{polly1998variability,biswas2019} and references within) in birds or mammals when a large sample is included.\\ 

  Parasitology & \textbf{Yes}. Males are commonly more susceptible to infection, have lower immune function, and higher parasite loads than females\cite{zuk2009sicker}. This is likely due to testosterone inhibiting the immune system\cite{Klein:2016aa}. However, this is not true for all species and all kinds of parasites, e.g. breeding female birds had more blood parasites than males\cite{mccurdy1998sex}. Differences in either direction may cause parasite load and diversity to be misrepresented where collections are sex biased.\\ 

  Stable isotope ecology & \textbf{Yes}. The demands of producing eggs, brooding, pregnancy, and lactation can alter stable isotope ratios \cite{fuller2004nitrogen}. Many species also have sex segregated diets, e.g. leopards\cite{voigt2018sex}, and foraging ranges, so stable isotope ratios may vary among sexes even in non-breeding individuals.\\ 

  Toxicology & \textbf{Yes}. As above, sexes may differ in foraging ecology, which has consequences for contaminant burden. Furthermore, females may be able to eliminate some contaminants via eggs (e.g. mercury\cite{robinson2012sex}), an option not available to males.\\ 
  Morphological evolution & \textbf{Yes}. There is extensive sexual dimorphism in many of the traits used in studies of morphological evolution, for example body size\cite{Uyeda15908}, thus tempo and mode of evolution may vary with sex.\\ 

  \hline
\label{table_consequences}
\end{longtable}


\end{landscape}


\section{Materials and Methods}
\subsection{Data collection and cleaning}
\subsubsection{Specimen data} 
We obtained museum bird and mammal collection records from the Global Biodiversity Information Facility (GBIF\cite{gbif}). 
Specifically we collated data from the American Museum of Natural History (AMNH; $n = 271,407$ records\cite{amnh-birds,amnh-mammals}), Field Museum of Natural History (FMNH; $n = 182,984$ records\cite{fmnh-birds,fmnh-mammals}), Mus\'{e}um National d'Histoire Naturelle (MNHN; $n = 86,126$ records\cite{mnhn-birds,mnhn-mammals}), National Museum of Natural History, Smithsonian Institution (NMNH; $n = 496,735$ records\cite{smithsonian-both}), and Natural History Museum, London (NHMUK; $n = 251,409$\cite{nhm-all}).
 All raw data can be downloaded from GBIF\cite{gbif}.

We obtained bird and mammal specimen records from the American Museum of Natural History (AMNH), the Field Museum of Natural History (FMNH), the Mus\'{e}um National d'Histoire Naturelle (MNHN), the National Museum of Natural History, Smithsonian Institution (NMNH), and the Natural History Museum, London (NHMUK). 
These specimens were obtained between 1751 and 2018, mostly through hunting or trapping, and sexed based on external genitalia or secondary sexual characters, for example plumage colouration or antlers. 
Note that we recognise that biological sex is a spectrum\cite{sciam2017}. 
We focus here on specimens identified as females and males for simplicity because there were very few recorded intersex specimens in collections databases, but we recognize the importance of these individuals.

The final dataset contained 2,496,328 specimen records (1,395,748 birds and 1,100,580 mammals), representing 17,348 species (12,574 birds and 4,774 mammals; note that some of these were synonymies). 
Of these, 20\% of bird specimens were female, 31\% were male, and 49\% were not sexed. 
For mammals, the number of non-sexed individuals was much lower at 15\%, likely because it is easier to identify sex in mammals, with 41\% female and 44\% male specimens. 
If we consider only sexed specimens, 40\% of bird and 48\% of mammal specimens were female (Figure A1). 
In real terms this represents 143,905 more male than female specimens in birds and 40,468 more male specimens in mammals. 
In the wild, adult sex ratios in many bird species are male skewed, though on average not as skewed as our results (n = 187 species, median 44.8\% female\cite{szekely2014sex}), however, 48\% is not a large deviation from the 50\% expected in many natural populations of mammals\cite{karlin1986theoretical}.


Prior to analyses we cleaned the data as follows. 
(i) Record type. To avoid confusing specimens with archives describing specimens we selected only preserved specimen records; 
(ii) Age. Juveniles can be harder to sex so we excluded all juveniles, young and foetuses from the dataset; 
(iii) Year. We removed invalid collection years (i.e. years later than 2018); 
(iv) Taxonomy. We removed subspecies names and used species binomials because we were  interested in species-level sex ratios. We corrected bird taxonomy using the GBIF backbone taxonomy\cite{gbif}, and mammal taxonomy using Mammal Species of the World\cite{wilson2005mammal}; 
(v) Type status. We split types into name bearing (Holotype, Syntype, Lectotype, Neotype) and non-name bearing (Paratype, Paralectotype, Paraneotype, Allotype, Topotype, Cotype) types. Where the records did not specify the kind of type we define these as ambiguous types; 
(vi) Sex. We standardized sex to either Female, Male or non-sexed, and removed all intersex or hermaphrodite individuals. 
We also excluded non-sexed individuals from the analyses; and finally 
(vii) we excluded all surplus data columns. 
The final dataset contained 2,496,611 specimens (1,395,748 birds and 1,100,863 mammals), representing 17,348 species (12,574 birds and 4,774 mammals). 

\subsubsection{Sexual dimorphism, plumage coloration and ornamentation data.}
We extracted median body masses (g) for males and females from Lislevand et al.\cite{lislevand2007avian} for birds and Jones et al.\cite{pantheria} for mammals, then calculate sexual size dimorphism by dividing mean male body mass by female body mass. 
The final dataset contains sexual size dimorphism data for 2,272 bird (18\% of species in our specimen dataset) and 1,314 mammal species (28\% of species in our specimen dataset). 
Note that the sample size here is lower because sex disaggregated body size data are rare.

To explore how ``showiness'' might influence sex bias, we include a measure of plumage coloration for passerine birds taken from Dale et al.\cite{dale2015data,dale2015effects}. 
This measure is based on the mean RGB (red green blue) values for 400 randomly chosen pixels in six patches (nape, crown, forehead, throat, upper breast, and lower breast) for each sex. 
We then calculated a plumage dimorphism score by dividing male plumage score by female score for each species. 
For mammals, we used the Handbook of Mammals of the World to identify mammals where males have ornamentation. We define ornamentation as horns, antlers and tusks. 
Where species had ornaments, we recorded whether both sexes or only males routinely possess them. The majority of species with ornaments were Artiodactyla (203 of 224 species). 
Other ornamented species were walruses, beaked whales, narwhals, rhinoceroses, okapis, and elephants, however only artiodactyls and the walrus were represented by more than 100 specimens in the dataset (see below), so were the only species to appear in our models.

The final cleaned data are available on the NHM Data Portal [link will be added on acceptance]. 

\subsection{Analyses}
All analyses investigated birds and mammals separately. 
We performed all analyses in R version 3.5.0\cite{R}. 
Reproducible scripts are available on GitHub at https://github.com/nhcooper123/sex-bias-museums [we will add a zenodo link on acceptance].

We first summarized the overall proportion of female specimens, and calculated the median proportion of females across species. 
In addition, we summarized differences in the proportion of female specimens across orders and types. 

Most species were represented by only a few specimens (Figure A8), with large skews towards either males or females at low numbers (Figure A9). 
To reduce problems this is likely to cause when fitting models, we use only species with 100 or more specimens in our models (see Supplementary Methods for more details), except in our change-through-time models. 
In these models our response variable is the proportion of males and females in each species for each year from 1880-2010 (before 1880 and after 2010 we do not have any species with sufficient specimens to include). 
As there were only 55 bird species and 1216 mammal species with over 100 specimens in a year, change-through-time models instead use all species with more than 50 specimens in a single year to increase the sample size ($n = 391$ birds, $n = 3428$ mammals).

We fitted all models using generalized linear models (GLM) with quasibinomial errors, with the proportion of female specimens (success) and the proportion of male specimens (failure) for each species as the response variable (i.e. a binomial response where the number of females and the number of males for each species were jointly modeled). 
Quasibinomial rather than binomial errors were used due to overdispersion (all models have deviance/residual degrees of freedom far greater than two; see output on GitHub for exact values), and we assessed the significance of model terms using Type II sums of squares. 
We used standard model checks for GLMs (Q-Q plot, histogram of residuals, residuals vs. linear predictors, response vs. fitted values) to assess model fit. 
We tested whether the proportion of female and male specimens varied with (i) orders; (ii) collection years (1880-2010); (iii) male body mass (log transformed); (iv) sexual size dimorphism (log transformed); (v) plumage dimorphism (log-transformed; passerine birds only); and (vi) ornamentation (mammals only). 




\newpage


Bias towards males in parts of natural history collections may have been driven by a number of reasons such as deliberate selection of males, either by collectors in the field or curators accessioning specimens, trapping biases (i.e. trapping method, season of collecting, conspicuous male behaviors or traits) or in some cases simply because there were more males in a population. 
Collectors may also actively avoid collecting females with young due to legislation, ethical or conservation considerations. 
To investigate these biases further, we estimated the proportion of female and male specimens for each species represented by more than 100 sexed individuals (see Materials and Methods and Supplementary Methods for an explanation of this cut-off). 
Overall the median proportion of female specimens in each species was 41\% in birds and 48\% in mammals. 
To test for correlates of bias we analyzed birds and mammals separately, and fitted all models using generalized linear models (GLM) with quasibinomial errors, with the proportion of female specimens (success) and the proportion of male specimens (failure) for each species as the response variable.

\section{Results and Discussion}

% figure 1
\begin{figure}
 \centering
  \includegraphics[width = \linewidth]{figures/orders-density-birds-six.png}
  \caption{Kernel density plots showing the \% female specimens in each species across the six largest orders of birds (from left to right, top to bottom: Passeriformes, Apodiformes, Piciformes, Psittaciformes, Charadriiformes, and Columbiformes). 
  Only species with at least 100 specimens are included. 
  The dashed line represents 50\% female specimens. 
  Silhouettes are from PhyloPic.org contributed by Ferran Sayol (parrot, hummingbird, tit), Steven Traver (woodpecker) and Alexandre Vong (shorebird).}
  \label{fig-bird_order_six}
\end{figure}

\subsection{Variation among orders.} 
The proportion of female specimens varied across orders for both birds ($F_{24, 1721} = 29.81$, $p < 0.001$; Figure \ref{fig-bird_order_six}; Figure A2; Table A1) and mammals ($F_{24, 1488} = 19.80$, $p < 0.001$; Figure \ref{fig-mammal_order_six}; Figure A3; Table A1). Most orders have more males than females (Table A1). 
In birds, of the 25 orders with sufficient data, only tinamous (Tinamiformes; 50.4\%) have more females, but these represent just four species in the dataset. 
The most male-biased orders with more than 25 species were pigeons and doves (Columbiformes; 36.8\% female), hummingbirds and swifts (Apodiformes; 37.2\%), and passerines (Passeriformes; 38.4\%). 
Adult sex ratios in Columbiformes and Passeriformes are generally male-skewed\cite{szekely2014sex,bosque2019skewed,mayr1939sex}, but hummingbirds are often female-skewed in the wild\cite{szekely2014sex,mayr1939sex}. 
This, along with evidence that, on average, Passeriformes are not as male biased as our results ($n = 54$ species, median 45.1\% female\cite{szekely2014sex}), suggests that greater availability of males alone cannot account for our results.

Seven of the 25 mammalian orders with sufficient data have more females, the most extreme being anteaters and sloths (Pilosa; 71.1\% female). 
Most mammal species have a sex ratio of 1:1 at birth\cite{karlin1986theoretical}, though this can vary in adults. 
Several species of sloth have higher numbers of females (up to 68.8\% females\cite{reyes2015informacion}) which may explain why we also find more females in collections, however, giant anteaters (\textit{Myrmecophaga tridactyla}) show strong male bias in the field (e.g.\cite{defreitas2015}), but strong female bias in collections (71.3\% female). 
Among the orders represented by more than 25 species in our data, only bats have more females (Chiroptera; 52.2\% female; Figure \ref{fig-mammal_order_six}), despite reportedly balanced adult sex ratios in the wild\cite{altringham}. 
This is likely related to widespread sex segregation in bat roosting sites, with many roosts containing individuals of only one sex\cite{altringham}. 
In the the past, bats were often trapped by collecting all individuals in a roost site, and female bats use fewer roost sites than males (e.g.\cite{encarnaccao2012spatiotemporal}), so skew towards females is not surprising. 
The most male-biased order of mammals were the even-toed ungulates (Artiodactyla; 39.7\% females), but although they exhibit a great deal of variation in adult sex ratio, on average, there are more females than males in wild populations\cite{berger1999sex} suggesting strong selection for male specimens in this order derived from the trophy hunting of large males that was common in the 19th and early 20th centuries.

% figure 2
\begin{figure}
 \centering
  \includegraphics[width = \linewidth]{figures/orders-density-mammals-six.png}
  \caption{Kernel density plots showing the \% female specimens in each species across the six largest orders of mammals (from left to right, top to bottom: Rodentia, Chiroptera, Soricomorpha, Carnivora, Primates, and Artiodactyla). 
  Only species with at least 100 specimens are included. 
  The dashed line represents 50\% female specimens. 
  Silhouettes are from PhyloPic.org contributed by Daniel Jaron (mouse), Yan Wong (bat), Becky Barnes (shrew), Lukasiniho (tiger), Sarah Werning (monkey), and Oscar Sanisidro (deer).
}
  \label{fig-mammal_order_six}
\end{figure}


\subsection{Changes through time.}
Conventional wisdom suggests that some male bias may be related to the age of the collections. 
We expected the bias to decrease towards the present due to changes in collection methods and motivations over the last century. 
To test this we modeled how the proportion of females for each species per year changed between 1880 and 2010. 
We found male bias increased for birds ($F_{1, 389} = 7.167$, $p = 0.008$; Figure \ref{fig-time}), but decreased for mammals ($F_{1, 3426} = 6.86$, $p = 0.009$; Figure \ref{fig-time}), however the effect sizes were extremely small (birds: $slope \pm SE = -0.002 \pm < 0.001$; mammals: $slope \pm SE = 0.001 \pm < 0.001$), indicating very little change in either class, i.e. there has been no improvement in the sex balance of collecting over the last 130 years.

% figure 3
\begin{figure}
 \centering
  \includegraphics[width = \linewidth]{figures/years-all.png}
  \caption{Changes in the proportion of female specimens for each species collected each year between 1880 and 2010. 
  Best fit lines and 95\% confidence intervals from quasibinomial generalized linear models are shown in grey. 
  Data points represent species with at least 50 specimens collected in a given year. 
  The dashed line represents 50\% female specimens.
}
  \label{fig-time}
\end{figure}

\subsection{Male body mass and sexual size dimorphism.}
A major suspected source of male bias in collections is deliberate selection for large, ``impressive'' male specimens. 
We found significant effects of male body size on the proportion of female specimens in both birds and mammals (Table A3), however, the direction and strength of the relationship varied among classes and orders (Figure \ref{fig-male-mass}; Figures A4-A5; Table A3). 
Bird species with larger males tended to have more female specimens, whereas the reverse was true for mammals. 
In mammals this was likely driven by a few orders with large males that have long been favored in collections (e.g. Artiodactyla, Carnivora) and have low median percentages of female specimens (Figure \ref{fig-mammal_order_six}, Table A1). 
Interestingly however, selection for males in these groups did not increase with increasing male body size (Figure A5), instead it appears male carnivores and artiodactyls were preferred over females, regardless of their body size.

% figure 4
\begin{figure}
 \centering
  \includegraphics[width = \linewidth]{figures/male-mass-all.png}
  \caption{Relationship between the percentage of female specimens for each species and log male body mass (g) for the species. 
  Only species with at least 100 specimens are included. 
  The dashed line represents 50\% female specimens. 
  Note that the x-axis scales are different for birds and mammals.
}
  \label{fig-male-mass}
\end{figure}

Rather than selecting large males \textit{per se}, collectors may favour males when the difference in size between females and males is large. 
To test this we calculated sexual size dimorphism by dividing male body mass by female body mass. 
We found that as sexual size dimorphism increased, i.e. as males became increasingly larger than females, there was more bias towards male specimens (Table A3), however, this result was entirely driven by differences among orders (Figure A4 and A5); when order was included in the models, sexual size dimorphism did not significantly influence specimen sex ratios over the effects of order (Table A3). 
As with body mass, this suggested certain orders were more likely to contain more male specimens, regardless of their size with respect to females, suggesting that other characteristics were driving their selection.

\subsection{Plumage and ornaments.} 
To further test whether collections were biased towards ``showy'' or ``impressive'' males we collated data on sexually dimorphic features in birds and mammals. 
For birds, we focused on passerines (Passeriformes) a group with highly sexually dimorphic plumage coloration in some species but not others, and for which there is detailed available data\cite{dale2015data,dale2015effects}. 
We calculated a sexual plumage dimorphism score for each species, and found that as males became increasingly more colorful than females, the proportion of female specimens decreased (Figure \ref{fig-plumage}; $F_{1, 828} = 58.95$, $p < 0.001$; $slope \pm SE = -0.416 \pm 0.054$). 
This relationship was not strong, but fits with anecdotal evidence of collections preferentially selecting, and displaying\cite{machin2008}, colorful male specimens, especially where plumage differences are large, for example in birds of paradise. 

% figure 5
\begin{figure}
 \centering
  \includegraphics[width = \linewidth]{figures/plumage.png}
  \caption{Relationship between the percentage of female specimens for each species and log plumage dimorphism scores in passerine birds. 
  Only species with at least 100 specimens are included. 
  The dashed line represents 50\% female specimens; the dotted line is the point at which males and females have the same plumage coloration. 
  Plumage dimorphism scores were calculated by dividing male plumage scores by female plumage scores (see Methods). 
}
  \label{fig-plumage}
\end{figure}

For mammals, we investigated ornamentation, which we defined as horns, antlers and tusks. 
Species with ornaments had significantly fewer female specimens than those without (Figure \ref{fig-horns}; $F_{1, 1510} = 47.11$, $p < 0.001$; $slope \pm SE = -0.344 \pm 0.051$). 
All bar one of the species (the walrus) with ornaments in our models were artiodactyls (other species with ornaments such as elephants have fewer than 100 specimens), and all artiodactyls have horns or tusks, suggesting that the strong male bias in Artiodactyla (39.7\% female; Figure \ref{fig-mammal_order_six}) was due to selection for males with these features. 
Within ornamented species there was no significant difference if both sexes or only males possessed the ornament (Figure \ref{fig-horns}; $F_{1, 58} = 0.501$, $p = 0.482$), suggesting that even where females are phenotypically different, preference is still given to collecting males. 
This is particularly concerning since most artiodactyl species are female skewed in the wild\cite{berger1999sex}. 

% figure 6
\begin{figure}
 \centering
  \includegraphics[width = \linewidth]{figures/ornamentation.png}
  \caption{Kernel density plots comparing the \% female specimens in each mammal species where ornamentations, i.e. horns or tusks, are present or absent (top panel), and when species have ornaments, whether these are found in both sexes or only males (bottom panel). 
  Only species with at least 100 specimens are included. 
  The dashed line represents 50\% female specimens. 
}
  \label{fig-horns}
\end{figure}

\subsection{Type specimens.} 
Perhaps our most notable finding focused on name bearing type specimens (holotypes, syntypes, lectotypes, and neotypes). 
Here the bias towards male specimens was extreme; only 25\% of bird and 39\% of mammal types were female (Figure A1). 
Although in some instances, males might be considered the appropriate sex for holotypes, we see no reason (other than limitations on specimens available) to not also designate a female paratype to represent the phenotypic range of a species. 
In mammals paratypes were almost 49\% female, but bird paratypes were 38\% female. 
Currently sex does not form any part of the International Code for Zoological Nomenclature (ICZN) recommendations for designating types, although some historical instructions for collectors emphasize the importance of multiple types (e.g.\cite{schuchert1897type}). 
Adding this to the ICZN is of vital importance moving forwards.

Museum professionals, and those using museum collections, should be aware that biases towards males exist in bird and mammal collections. 
These biases may have implications for studies that assume their samples are representative. Indeed almost all major uses of museum collections, as detailed in Table \ref{table_consequences}, may be strongly influenced by sex biases. 
It is likely that not all male bias was the result of active selection by collectors or curators; there may also be bias towards males where collecting one sex is more likely due to some sex-biased trait. 
For example, higher dispersal in males may result in males being more likely to come into contact with hunters or traps, for example when conspecific agonistic territorial behavior (e.g. calls) is used to bring individuals towards a trap. 
Males may also exhibit lower levels of neophobia increasing their likelihood of being captured though evidence for this is limited\cite{crane2017patterns,laviola1992sexual}. 
Our analyses, however, suggest that much of the male bias in collections is the result of historical active selection of males, particularly where males are the larger (for mammals at least), ``showier'' sex. 
To reduce these imbalances, collectors in the field should strive to avoid trapping biases and biases in selecting individuals to collect. 
In natural history museums, curators and collections managers should have an awareness of the biases within their collections (not just in terms of sex but also in terms of age, locality, and other factors), and attempt to acquire material to best resolve those biases, whatever their cause. 
Natural history collections play a critical role in informing multiple research disciplines answering vital questions for the future of biodiversity\cite{meineke2018biological} and are also key resources for public engagement and interaction with biodiversity\cite{machin2008}. 
Therefore it is paramount that we continue developing these resources while using a more comprehensive and better informed approach. 
Finally, researchers investigating broad-scale variation in species should account for these biases when designing data collection protocols and/or in downstream analyses and declare how they dealt with those biases in resulting publications. 
Our analyses place particular pressure on taxonomists to think more carefully about sex when defining name-bearing types, and suggest more designation of opposite sex paratypes would be desirable, particularly in birds. 




\section{Acknowledgments}
We thank all collections managers for inputting their specimen data.

% References
\bibliographystyle{vancouver}
\bibliography{sex-bias}

%\section{Author contributions}
%NC performed the analyses and wrote the first draft. All authors contributed to study design, interpreted results, revised the manuscript, and approved the submitted version.

%\section{Competing interests}
%The authors declare no competing interests.

\section{Data availability}\label{data-code-and-materials}
Data are available from the NHM Data Portal (link will be added) and GBIF. 
R code is available from GitHub (https://github.com/nhcooper123/sex-bias-museums)(Zenodo DOI will be added on acceptance).

\end{document}